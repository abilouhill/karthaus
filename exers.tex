% Copyright 2009--2010  Ed Bueler

\documentclass[11pt,final]{amsart}
\addtolength{\topmargin}{-0.15in}
\addtolength{\oddsidemargin}{-.55in}
\addtolength{\evensidemargin}{-.55in}
\addtolength{\textwidth}{1.2in}
\addtolength{\textheight}{0.3in}

\renewcommand{\baselinestretch}{1.05}

\usepackage{bm,url,xspace}
\usepackage{verbatim}
\usepackage{fancyvrb}
\usepackage{amssymb,amsmath}
\usepackage[final,pdftex]{graphicx}
\usepackage[pdftex]{hyperref}

\newcommand{\ddt}[1]{\ensuremath{\frac{\partial #1}{\partial t}}}
\newcommand{\ddx}[1]{\ensuremath{\frac{\partial #1}{\partial x}}}
\newcommand{\ddy}[1]{\ensuremath{\frac{\partial #1}{\partial y}}}
\newcommand{\pp}[2]{\ensuremath{\frac{\partial #1}{\partial #2}}}
\newcommand{\eps}{\epsilon}
\newcommand{\RR}{\mathbb{R}}

\newcommand{\bbF}{\mathbf{F}}
\newcommand{\bU}{\mathbf{U}}
\newcommand{\bY}{\mathbf{Y}}
\newcommand{\bb}{\mathbf{b}}
\newcommand{\bw}{\mathbf{w}}

\newcommand{\grad}{\nabla}
\newcommand{\Div}{\nabla\cdot}
\newcommand{\strainrate}{D}
\newcommand{\devstress}{\tau}

\newcommand{\exer}[2]{\medskip\noindent \textbf{#1.}\quad #2}
\newcommand{\mname}[1]{\href{http://www.dms.uaf.edu/~bueler/karthaus/mfiles/#1}{\texttt{#1}}}

\newcommand{\Matlab}{\textsc{Matlab}\xspace}
\newcommand{\Octave}{\textsc{Octave}\xspace}


\begin{document}

\title{Exercises to accompany \\ ``Numerical modelling of ice sheets and ice shelves''}

\date{\today.  Copyright 2009--2010  Ed Bueler}

\maketitle


\section*{Exercises for section ``heat analogy \& numerics''}

\exer{I.1}{On the slide titled ``compare to heat equation'' there is a note saying ``where $\tilde \mu$ is a material constant proportional to $\Delta x^{-2}$''.  Explain why this constant has this proportionality.}

\exer{I.2}{Fill in the details on the slides titled ``finding similarity solution of heat equation'', to derive the Green's function of the heat equation.}

\exer{I.3}{Assume $u$ has continuous second derivatives.  Show
  $$u_x(t,x) = \frac{u(t,x+\Delta x) - u(t,x-\Delta x)}{2\Delta x} + O(\Delta x^2).$$
}
\exer{I.4}{This is an easy exercise in \Matlab/\Octave colon notation, motivated by the slide which displays the code \mname{heat.m}:
\renewcommand{\labelenumi}{(\alph{enumi})}
\begin{enumerate}
\item Give colon notation for the list of numbers $4,7,\dots,100$.  Give colon notation that extracts columns $4,7,\dots,100$ of a $100 \times 100$ arbitrary matrix $A$ to form a $100 \times 33$ matrix $B$.
\item Rewrite \mname{heat.m} \emph{without} any colon notation.
\end{enumerate}}

\exer{I.5}{\emph{This extended exercise concerns the numerical treatment of ``$\Div\left(D\,\grad\right)$''.} 
\renewcommand{\labelenumi}{(\alph{enumi})}
\begin{enumerate}
\item Show that if $D=D(x,y)$ and $u=u(x,y)$ then $\Div \left(D\, \grad u\right) = D \grad^2 u + \grad D \cdot \grad u$.
\item Write down the obvious $O(\Delta t)+O(\Delta x^2)$ explicit finite difference method for the equation $u_t = D_0 u_{xx} + E_0 u_x$, assuming $D_0>0$ and $E_0$ are constant.  (\emph{You will need to use a centered approximation for $u_x$.})
\item The explicit method for the heat equation is solved for the unknown $u_j^{n+1}$.  Do the same for your explicit method from part (b). 
\item Stability for your method will occur if the right hand side from part (c) has all positive coefficients.  If $|E_0| \gg D_0$, what does this say about $\Delta t$?
\item Why we keep ``$\Div\left(D\,\grad\right)$'' in that form?  How is this related to staggered grid?
\end{enumerate}
}


\bigskip
\section*{Exercises for section ``shallow ice sheets''}

\exer{II.1}{Is P.~Halfar male or female, and what does the ``P.'' stand for?}

\exer{II.2}{The simplest model of earth deformation is to assume \emph{instantaneous pointwise isostasy}, which says that the bed is depressed by a fixed fraction $0<f<1$ of the ice thickness.  Include this effect in the SIA model which, namely equation (1) in the lecture.  Also include it in the Halfar solution, and plot the results.}

\exer{II.3}{Nye [2000] derives the Halfar solution differently from how I have done it here.  Read, and explain the principles behind, Nye's nicer version.}

\exer{II.4}{Find and read [Mahaffy, 1976].  It sets a standard for a quality application of a model for flowing ice.}

\exer{II.5}{On the slided titled ``verifying our SIA code \texttt{siaflat.m}'' we see that the numerical error in ice thickness near the grounded margin of an ice sheet is much larger than elsewhere.  Why?  How general is this situation?}

\exer{II.6}{On the slide titled ``kinematic and mass continuity equations 2'' we show that (3) \& (4) $\implies$ (5).  Now show that (4) \& (5) $\implies$ (3).  That is, get the surface kinematical equation from the mass continuity and basal kinematical equations.}

\exer{II.7}{
\renewcommand{\labelenumi}{(\alph{enumi})}
\begin{enumerate}
\item Show $h^*_{x^*} = (\ell/d) h_x$ in the scalings given on the slides titled ``scaling the variables''.
\item On the slide titled ``finally get to the SIA'' we have ``$\tau_{13} = - (h - z) h_x$.''  Re-dimensionalize it.  That is, put stars on and then go back to the original variables.
\end{enumerate}
}


\bigskip
\section*{Exercises for section ``Shallow shelves and streams''}

\exer{III.1}{\renewcommand{\labelenumi}{(\alph{enumi})}
\begin{enumerate}
\item Find one reference on, and one actual code for solving, tridiagonal systems. \emph{Now throw them away}.  A decent matrix solver will recognize the sparsity pattern and act accordingly, so you don't have to.
\item What sparse structure will the SSA matrix have in two spatial variables is we use a finite difference method and square grid cells?
\end{enumerate}}

\exer{III.2}{What is going on in \mname{flowline.m} (slide titled ``solving the inner linear problem 4'') with the lines about the ``\texttt{scale}'' variable?  What happens if these lines are commented out?}

\exer{III.3}{On the slide titled ``the full monty, with a grounding line'' we display a reasonably complete continuum model for an ice stream crossing the grounding line and flowing into an ice shelf.  (In the flowline case.) \renewcommand{\labelenumi}{(\alph{enumi})}
\begin{enumerate}
\item Modify \mname{ssaflowline.m} to include the term ``$- C|u|^{m-1}u$'', first in the $m=1$ case of linear till.  (The MISMIP project can supply a value of $C$.)  Does this term modify the matrix or the right hand side, in the matrix problem solved by \mname{flowline.m}?
\item Run the example on an ice stream geometry of your choice.
\end{enumerate}}



% some slides that are too much for lecture but should be in exercises:

\begin{comment}

\subsection{derivation of SSA}

\begin{frame}{derivation of SSA from Stokes; plane flow case}

\begin{itemize}
\item following appendix A of Schoof [2006]\nocite{SchoofStream}
\item starting point is the same set of $n=3$ plane flow Stokes equations as used in the derivation of the SIA
\item recall the equations, including upper surface boundary conditions, but with flow law in ``viscosity form''; $B=A^{-1/3}$:
\small
\begin{align*}
u_x + w_z &= 0 \\
p_x &= \tau_{11,x} + \tau_{13,z} \\
p_z &= \tau_{13,x} - \tau_{11,z} - \rho g \\
\tau_{11} &= B d^{-2/3} u_x \\
\tau_{13} &= B d^{-2/3} \frac{1}{2}\left(u_z + w_x\right) \\
\tau_{13}\big|_h &= (\tau_{11}\big|_h - p\big|_h) h_x \\
p|_h + \tau_{11}|_h + \tau_{13}|_h h_x &= 0
\end{align*}
\normalsize
\item (mass continuity equation not considered in this derivation)
\end{itemize}
\end{frame}


\begin{frame}{derivation of SSA from Stokes 2}

\begin{itemize}
\item merely \emph{different scalings} and \emph{different basal boundary conditions} from SIA derivation
\item ice shelf (and not ice stream) case here
\item the basal boundary experiences no \emph{traction} from ocean water, in an ice shelf, but there is significant \emph{pressure}
\item ice shelves are floating so $\rho H = - \rho_w b$
\item the basal stress condition is, therefore,
	$$\sigma_{ij} n_j = \rho g H n_i$$
where $\textbf{n}$ is normal to base
\item using $\mathbf{n} = (-b_x,0,1)$, get two basal conditions
\begin{align*}
\tau_{13}\big|_b &= (\tau_{11}\big|_b + \rho g H - p\big|_b) b_x \\
p|_b + \tau_{11}|_b + \tau_{13}|_b b_x &= \rho g H
\end{align*}
\end{itemize}
\end{frame}


\begin{frame}{derivation of SSA from Stokes 3}

%\mu = d case where \delta=1
\begin{itemize}
\item scalings of variables similar to SIA:
\begin{align*}
x &= \ell\, x^* &         u &= [u] u^* \\
z &= d\, z^*    &         w &= \eps [u] w^* \\
h &= d\, h^*    & \tau_{11} &= [\tau] \tau_{11}^* \\
H &= d\, H^*    & \tau_{13} &= \eps [\tau] \tau_{13}^* \\
b &= d\, b^*  & p - \rho g (h-z) &= [\tau] p^*
\end{align*}
\item \dots but note that scalings of pressure $p$ and stress components $\tau_{11},\tau_{13}$ different from SIA case
\item scale relations:
  \begin{align*}
  \eps &= \frac{d}{\ell}           & [\tau] &= \rho g d  &
  \frac{[u]}{\ell} &= [A] [\tau]^3
  \end{align*}
\end{itemize}
\end{frame}


\begin{frame}{derivation of SSA from Stokes 4}

\begin{itemize}
\item re-written Stokes equations, in new non-dimensional coordinates, but without stars on variables:
\begin{align*}
u_x + w_z &= 0 \\
p_x &\stackrel{\ast}{=} \tau_{11,x} + \tau_{13,z} - h_x \\
p_z &= -\tau_{11,z} + \eps^2 \tau_{13,x} \\
\tau_{11} &= d^{-2/3} u_x \\
u_z &\stackrel{\ast\ast}{=} 2 \eps^2 (\tau_{13} - w_x) \\
\tau_{13}\big|_h &\stackrel{\dagger}{=} (\tau_{11}\big|_h - p\big|_h) h_x \\
p|_h + \tau_{11}|_h &= -\eps^2 h_x \\
\tau_{13}\big|_b &\stackrel{\dagger}{=} (\tau_{11}\big|_b - p\big|_b) b_x \\
p\big|_b + \tau_{11}\big|_b &= - \eps^2 b_x \tau_{13}\big|_b
\end{align*}
\item here $d^2 = u_x^2 + \eps^2 (1/4) \left(2 (\tau_{13} - w_x) + w_z\right)^2$
\end{itemize}
\end{frame}


\begin{frame}{derivation of SSA from Stokes 5}
\label{slide:getssa}

\begin{itemize}
\item to get to SSA equations, first take system of equations on previous page and drop $\eps^2$ terms
\item then integrate equation $\ast$ from base to surface, use the equations marked $\dagger$ to evaluate $\tau_{13}$ at base and surface, and use Leibniz rule to get
	$$\left(\int_b^h p\,dz\right)_x \stackrel{\Omega}{=} \left(\int_b^h \tau_{11}\,dz\right)_x - H h_x$$
\item equation marked $\ast\ast$ shows that $u$ is independent of $z$, but then it follows from expression for $d^2$ and other equations that:
	$$u,d,\tau_{11}, \text{ and } p \text{ are all independent of } z$$
\item also $p=-\tau_{11}$, so, from above equation marked $\Omega$,
	$$\left(2 H \tau_{11}\right)_x - H h_x = 0$$
\item re-dimensionalize this to get SSA equation for velocity in ice shelf case,
  $$\left(2 A^{-1/3} H |u_x|^{-2/3} u_x\right)_x - \rho g H h_x = 0$$
\end{itemize}
\end{frame}

\begin{frame}{exers to go with above derivation}
\exer{III.n}{Fill in the details on the integrals in the second bullet point on the previous slide.}

\exer{III.n+1}{Do the re-dimensionalization step on the previous slide, the last bullet point.}
\end{frame}



\bigskip
\section*{Exercises for section ``free boundary problems''}

\exer{IV.1}{Modify \mname{plothalfar.m} to show surfaces of $H^{8/3}$ instead of $H$.  Compare to pictures on slide 35.  Relate to the material on free boundary problems below.
}

\end{comment}

\end{document}

