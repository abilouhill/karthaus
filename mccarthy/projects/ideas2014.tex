\documentclass[11pt,final]{amsart}%default 10pt
%prepared in AMSLaTeX, under LaTeX2e

\usepackage[total={6.2in,9.0in},top=1.2in,left=1.1in]{geometry}

\usepackage{natbib}

\usepackage{amssymb,alltt,verbatim,xspace,fancyvrb,color,empheq}
\usepackage{palatino}
\usepackage[sc]{mathpazo}
\usepackage[T1]{fontenc}

% check if we are compiling under latex or pdflatex
\ifx\pdftexversion\undefined
  \usepackage[final,dvips]{graphicx}
\else
  \usepackage[final,pdftex]{graphicx}
\fi

% hyperref should be the last package we load
\usepackage[pdftex,
                colorlinks=true,
                plainpages=false, % only if colorlinks=true
                linkcolor=blue,   % only if colorlinks=true
                citecolor=black,   % only if colorlinks=true
                urlcolor=magenta     % only if colorlinks=true
]{hyperref}

\newcommand{\normalspacing}{\renewcommand{\baselinestretch}{1.05}\tiny\normalsize}
\newcommand{\tablespacing}{\renewcommand{\baselinestretch}{1.0}\tiny\normalsize}
\normalspacing

% math macros
\newcommand\bG{\mathbf{G}}
\newcommand\bv{\mathbf{v}}
\newcommand\bV{\mathbf{V}}
\newcommand\bq{\mathbf{q}}
\newcommand\bQ{\mathbf{Q}}

\newcommand\CC{\mathbb{C}}
\newcommand{\DDt}[1]{\ensuremath{\frac{d #1}{d t}}}
\newcommand{\ddt}[1]{\ensuremath{\frac{\partial #1}{\partial t}}}
\newcommand{\ddx}[1]{\ensuremath{\frac{\partial #1}{\partial x}}}
\newcommand{\ddy}[1]{\ensuremath{\frac{\partial #1}{\partial y}}}
\newcommand{\ddxp}[1]{\ensuremath{\frac{\partial #1}{\partial x'}}}
\newcommand{\ddz}[1]{\ensuremath{\frac{\partial #1}{\partial z}}}
\newcommand{\ddxx}[1]{\ensuremath{\frac{\partial^2 #1}{\partial x^2}}}
\newcommand{\ddyy}[1]{\ensuremath{\frac{\partial^2 #1}{\partial y^2}}}
\newcommand{\ddxy}[1]{\ensuremath{\frac{\partial^2 #1}{\partial x \partial y}}}
\newcommand{\ddzz}[1]{\ensuremath{\frac{\partial^2 #1}{\partial z^2}}}
\newcommand{\Div}{\nabla\cdot}
\newcommand\eps{\epsilon}
\newcommand{\grad}{\nabla}
\newcommand{\ihat}{\mathbf{i}}
\newcommand{\ip}[2]{\ensuremath{\left<#1,#2\right>}}
\newcommand{\jhat}{\mathbf{j}}
\newcommand{\khat}{\mathbf{k}}
\newcommand{\nhat}{\mathbf{n}}
\newcommand\lam{\lambda}
\newcommand\lap{\triangle}
\newcommand\Matlab{\textsc{Matlab}\xspace}
\newcommand\RR{\mathbb{R}}
\newcommand\vf{\varphi}

\newcommand{\Wlij}{W^l_{i,j}}
\newcommand{\Wij}{W_{i,j}}
\newcommand{\Plij}{P^l_{i,j}}
\newcommand{\Pij}{P_{i,j}}
\newcommand{\Ylij}{Y^l_{i,j}}
\newcommand{\Yij}{Y_{i,j}}
\newcommand{\upp}[3]{\big<#1\big|_{#3}\,#2\big>}

\newcommand{\Nbreen}{Nordenski\"oldbreen\xspace}


\title[]{Notes on 2014 McCarthy projects}

\author[]{Ed Bueler}


\begin{document}
%\graphicspath{{../figs/}}

\scriptsize \hfill \today \normalsize
\vspace{0.5in}

\maketitle
\thispagestyle{empty}

\setcounter{equation}{19}

\section*{PROJECT N: Models for hydrology and sliding on the Kennicott glacier}

\begin{quote}
\noindent ADVISOR: Ed Bueler

\medskip
\noindent DESCRIPTION: We will follow \cite{Bartholomausetal2011} in building a "lumped" numerical model of the hydrology and sliding of the Kennicott glacier during summer conditions, especially considering the sliding generated by the annual flood from the drainage of Hidden Creek Lake.  Their model involves ordinary differential equations and is well-constrained by data.  Then we'll look at either inverting the model for subglacial parameters or extending the model to a flowline where different parts of the glacier can respond at different rates.

\medskip
\noindent SOFTWARE REQUIREMENTS: Matlab or Octave

\medskip
\noindent REQUIRED STUDENT BACKGROUND: Minimal exposure to differential equations and use of Matlab or similar.
\end{quote}

\subsection*{Details} Use these equations from my Correspondence \citep{Bueler2014correspondence}:
\begin{equation}
\frac{dA_c}{dt} = u_b h + Z - C_c A_c (P_o-P)^n.  \label{eq:barth:cavityevolution}
\end{equation}
\begin{equation}
\frac{dP}{dt} = \frac{\rho_w g}{L W \phi} \left(Q_{in}(t) - Q_{out}(t) - \frac{f L }{h} \frac{d A_c}{dt}\right).
\end{equation}
This ODE system can be treated as
\begin{equation}
M \frac{d\bv}{dt} = G(t,\bv) \label{eq:system}
\end{equation}
where $M$ is lower triangular.  Note melt $Z$ is determined by (4) in \cite{Bartholomausetal2011} (=Barth) from geometric and/or known constants and $Q_{out}$; i.e.~$Z=\alpha Q_{out}$ where $\alpha$ is constant.

Note the basic mass conservation idea
\begin{equation}
\frac{dS}{dt} = Q_{in}(t) - Q_{out}(t). \label{eq:barth:massconserve}
\end{equation}
Additionally $P = ((\rho_w g)/(LW\phi)) S_{en}$ and $S_{sub} = (fLW/h) A_c$ so $S = S_{en} + S_{sub}$.  Thus, given $A_c$ and $P$ we can recover $S_{en},S_{sub},S$.  One could instead use ODE system \eqref{eq:barth:cavityevolution} and \eqref{eq:barth:massconserve} with $P$ in \eqref{eq:barth:cavityevolution} written as $P= c_1 S_{en} = c_1 (S - S_sub) = c_1 (S - c_2 A_c)$; this is closer to what Barth did.

To determine sliding, note equation (1) in Barth gives $z_w$ from $S_{en}$, thus $P$, and then equation (2) from Barth gives the effective pressure $N$.  Then equation (3) in Barth gives sliding velocity $u_b$.  This can all be post-processing on solution to ODE system.

Use ODE solver to solve system \eqref{eq:system} above.  Synthetic case from Figure 6 in Barth first.  Then use data (FIXME: which I need to get from Tim) and get Figure 7 in Barth.

\subsection*{Extension 1}  From formula (4) in Barth for $Z$, it makes sense, given all that is unknown subglacially, to write equation \eqref{eq:barth:cavityevolution} as
\begin{equation*}
\frac{dA_c}{dt} = \alpha u_b + \beta Q_{out} - C_c A_c (P_o-P)^n.
\end{equation*}
Then one could do inversion for $\alpha,\beta$ to fit data from Figure 8.  Roughly $\alpha$ is scale for cavitation and $\beta$ is scale for melting.

\subsection*{Extension 2}  From Correspondence \citep{Bueler2014correspondence} we have a PDE for a distributed flowline version:
\begin{equation}
\frac{\phi W}{\rho_w g} \frac{\partial P}{\partial t} = - \frac{\partial Q}{\partial x} - \frac{f}{h} \left[u_b h + Z - C_c A_c (P_o-P)^n\right]. \label{eq:barth:distpressure}
\end{equation}
Using this requires speculation on Darcy flux.


\bigskip\bigskip
\section*{Project N+1: From ice shelves in the lab to good numerical models}

\begin{quote}
\noindent ADVISOR: Ed Bueler

\medskip
\noindent DESCRIPTION:  One of the keys to effective modeling of the Antarctic ice sheet is capturing the way grounding lines move.  New laboratory models using  glycerine and salt water give moving grounding lines which can be decently instrumented.  For the high-lateral-confinement case we will build simple numerical schemes for the simple equations which are known to capture the dynamics.  Unlike much of the speculative literature of numerical grounding line modeling, we'll have the chance to be actually wrong (or even right, within known misfit). 

\medskip
\noindent SOFTWARE REQUIREMENTS: Matlab or Octave

\medskip
\noindent REQUIRED STUDENT BACKGROUND: Minimal exposure to differential equations and use of Matlab or similar.
\end{quote}

\subsection*{Details} FIXME  \cite{PeglerListerWorster2012,PeglerWorster2012,SayagPeglerWorster2012} but especially \cite{Pegleretal2013}

\small
\bibliography{ice-bib}  % generally requires link to pism/doc/ice_bib.bib
\bibliographystyle{agu}
\normalsize



\end{document}
