%kartbook.tex      
%karthausbook

%preamble
  
\documentclass[12pt]{report}
\usepackage{graphics}
\usepackage{graphicx}
\usepackage{epstopdf}
\usepackage{amsmath}
\usepackage{amssymb}
\usepackage{makeidx}


% NEW FROM OLAF
\usepackage{epsfig}
\usepackage{natbib}
\usepackage{rotating,subfigure}

% NEW FROM BUELER
\usepackage{verbatim,empheq,fancyvrb}
\usepackage[dvipsnames]{xcolor}


\topmargin=-0.2in
\textheight=8.4in
\oddsidemargin=0.1in
\textwidth=6in

%choose one or other:
%\newcommand{\lab}[1]{\label{#1} \quad #1\quad}%annotates equations
%
\newcommand{\lab}[1]{\label{#1}}
%%%%%%
\newcommand{\ben}{\begin{enumerate}}
\newcommand{\een}{\end{enumerate}}
\newcommand{\be}{\begin{equation}}
\newcommand{\ee}{\end{equation}}
\newcommand{\bdm}{\begin{displaymath}}
\newcommand{\edm}{\end{displaymath}}
\newcommand{\bea}{\begin{eqnarray}}
\newcommand{\eea}{\end{eqnarray}}
\newcommand{\bean}{\begin{eqnarray*}}
\newcommand{\eean}{\end{eqnarray*}}
\newcommand{\bmat}[1]{\left(\begin{array}{#1}}
\newcommand{\emat}{\end{array}\right)}
\newcommand{\bm}[1]{\mbox{\boldmath$ #1 $}}
\newcommand{\tab}{\hspace{0.8in}}
\newcommand{\no}{\noindent}
\newcommand{\non}{\nonumber}
\newcommand{\pvint}{-\hspace{-1.1em}\int}%ordinary pv int after symbol
\newcommand{\bpvint}{-\hspace{-1em}\int}%pvint after bracket
\newcommand{\marg}{\marginpar}
\newcommand{\real}{\mbox{Re\,}}
\newcommand{\realnos}{\mbox{{\bf R}}}
\newcommand{\complexnos}{\mbox{{\bf C}}}
\newcommand{\integers}{\mbox{{\bf Z}}}
\newcommand{\naturalnos}{\mbox{{\bf N}}}
\newcommand{\rationalnos}{\mbox{{\bf Q}}}
\newcommand{\mixpd}[3]{\frac{\partial ^{2} {#1}}{\partial{#2}\partial{#3}}}
\newcommand{\pdtwo}[2]{\frac{\partial ^{2} {#1}}{\partial{#2}^{2}}}
\newcommand{\pdone}[2]{\frac{\partial {#1}}{\partial {#2}}}
\newcommand{\sdone}[2]{\frac{d {#1}}{d {#2}}}
\newcommand{\sdtwo}[2]{\frac{d^2 {#1}}{d {#2}^2}}
\newcommand{\sDone}[2]{\frac{D {#1}}{D {#2}}}
%\newcommand{\tfrac}[2]{\textstyle{\frac{#1}{#2}}}
%\newcommand{\dfrac}[2]{\displaystyle{\frac{#1}{#2}}}
\newcommand{\sfrac}[2]{\small{\frac{#1}{#2}}}
\newcommand{\cosec}{\mbox{cosec\,}}
\newcommand{\imag}{\mbox{Im\,}}

\newcounter{thequestion}
\newcounter{question}
\newcommand{\resetquestnos}[1]
			{\renewcommand{\thequestion}{#1.\arabic{question}}
	\setcounter{question}{0}}
\newcommand{\reseteqnos}[1]{\renewcommand{\theequation}{#1.\arabic{equation}}
	\setcounter{equation}{0}}
\newcommand{\resetfignos}[1]{\renewcommand{\thefigure}{#1.\arabic{figure}}
	\setcounter{figure}{0}}
\newcommand{\resetsectnos}[1]{\renewcommand{\thesection}{#1.\arabic{section}}
	\setcounter{section}{0}}
\newcommand{\resetchapno}[1]{\renewcommand{\thechapter}{#1}
	\setcounter{chapter}{0}}

%begin exercises:
\newcommand{\bex}{\section*{Exercises}
\addcontentsline{toc}{section}{\protect\numberline{}{Exercises}}
\begin{list}%
{\arabic{chapter}.\arabic{question}}{\usecounter{question}}
}
\newcommand{\eex}{\end{list}}

%begin appendix exercises: syntax \beax{appref} \eeax

\newcommand{\beax}[1]{\section*{Exercises}
\addcontentsline{toc}{section}{\protect\numberline{}{Exercises}}
\begin{list}%
{#1.\arabic{question}}{\usecounter{question}}
}
\newcommand{\eeax}{\end{list}}


\newcommand{\react}[2]{\mbox{ \raisebox{-2ex} %reversible reaction arrows
{$\stackrel   %3 on top of 1
       {\stackrel   %2 on top of 1
             { \stackrel{  
            \mbox{\raisebox{0.6ex}{ ${\scriptstyle #1}$ }} }
                          {\textstyle \rightleftharpoons } }
             {#2                                           } }
       {
                                                             }$} }}
\newcommand{\rightreact}[1]{\stackrel{
\mbox{\raisebox{0.6ex}{ ${\scriptstyle #1}$} }}{\rightarrow}}

\newcommand{\hemisphere}{
\begin{picture}(8,10)(-1,-1)
\put(0,1){\line(1,0){8}}
\put(4,1){\oval(8,8)[t]}
\end{picture}
}
\newcommand{\sphere}{_{\small{\bigcirc}}}

\def\Ai{\mbox{Ai}}
\def\erfc{\mbox{erfc\,}}
\def\erf{\mbox{erf\,}}
\def\sech{\mbox{sech\,}}
\def\gl{ \mbox{\ \raisebox{-.9ex}{$\stackrel{\textstyle <}{>}$}}\ }
\def\lg{ \mbox{\ \raisebox{-.9ex}{$\stackrel{\textstyle >}{<}$}}\ }
\def\gsim{ \mbox{ \raisebox{-.9ex}{$\stackrel{\textstyle >}{\sim}$} } }
\def\lsim{ \mbox{ \raisebox{-.9ex}{$\stackrel{\textstyle <}{\sim}$} } }
\def\sgn{\mbox{sgn\,}}
\def\dsp{\displaystyle}
\def\txt{\textstyle}
\def\eps{\varepsilon}
\def\={&=&}
\def\det{{\rm det}\,}
\def\tr{{\rm tr}\,}
\def\dnon{\dfrac{}{}\nonumber\\}


\newcommand{\pd}{\partial}
\newcommand{\divg}{{\bm \nabla}.\,}
\newcommand{\del}{{\bm \nabla}}
\newcommand{\grad}{{\bm \nabla}\,}
\newcommand{\curl}{{\bm \nabla}\times}
\newcommand{\udotgrad}{{\bf u}\,.\!{\bm \nabla}}
%
%specific here
%
\newcommand{\sat}{{\rm sat}}
\newcommand{\Pe}{{\rm Pe}}

\newcommand{\vect}[1]{{\bf #1}}
\newcommand{\rom}[1]{{\rm #1}}
%\newcommand{\eqref}[1]{(\ref{#1})}

%chemical:
\def\ox{O$_2$}
\def\co2{CO$_2$}
\def\h2o{H$_2$O}
\def\sio2{SiO$_2$}
\def\mg{{\rm Mg}}
\def\si{{\rm Si}}
\def\fe{{\rm Fe}}
\def\oo{{\rm O}}

%subscripts and superscripts:

\def\sub{^{}_}
\def\super{_{}^}


%fractions
\def\half{\tfrac{1}{2}}
\def\quarter{\tfrac{1}{4}}
\def\third{\tfrac{1}{3}}
\def\sixth{\tfrac{1}{6}}
\def\ninth{\tfrac{1}{9}}
\def\12th{\tfrac{1}{12}}
\def\24th{\tfrac{1}{24}}
\def\48th{\tfrac{1}{48}}

\def\3.4ths{\tfrac{3}{4}}

\def\twothirds{\tfrac{2}{3}}
\def\threequarters{\tfrac{3}{4}}

%units:
\newcommand{\up}[1]{$^{{#1}}$}

%punctuation
\def\ie{i.\,e.,\ }
\def\eg{e.\,g.,\ }
\def\etal{{\it et al}.\ }
\def\dis{\dfrac{}{}}

%appendices
\def\apA{A } % projects
\def\apB{B } % excursions

\hyphenation{Helm-holtz}

\title{Glaciers and ice sheets in the climate system\\ 
\vspace*{0.8in} \large {The Karthaus summer school lecture notes}\vspace*{0.8in} }
\author{Edited by A.\,C.\ Fowler, G.\,H.\ Gudmundsson and A.\ Jenkins\vspace*{0.4in} }
\date{\today}

\makeindex

%\nofiles
% not to produce updated toc file\includeonly{kartpref,kart1,kart2,kart3,kartrefs}
\includeonly{kart8,kartrefs}

\begin{document}

\maketitle


\pagenumbering{roman}

\tableofcontents
\include{contrib}
\include{kartpref}%
\pagenumbering{arabic}

\setcounter{chapter}{0}

%chapter 1 flow of ice/Gudmundsson/ACF
\setcounter{subsection}{0}
\reseteqnos{1}
\include{kart1}%ch1

%chapter 2 thermal structure/?/ACF
\setcounter{subsection}{0}
\reseteqnos{2}
\include{kart2}%ch2

%chapter 3 basal processes/Fowler/GHG
\setcounter{subsection}{0}
\reseteqnos{3}
\include{kart3}%ch3

%chapter 4 tidewater glaciers/?+Howat/AJ
\setcounter{subsection}{0}
\reseteqnos{4}
\include{kart4}%ch4

%chapter 5 ice shelf--ocean interaction/Jenkins/GHG
\setcounter{subsection}{0}
\reseteqnos{5}
\include{kart5}%ch5

%chapter 6 polar meteorology/van den Broeke/Reijmer
\setcounter{subsection}{0}
\reseteqnos{6}
\include{kart6}%ch6

%chapter 7
\setcounter{subsection}{0}
\reseteqnos{7}
\include{kart7}%ch7

%chapter 8 
\setcounter{subsection}{0}
\reseteqnos{8}
%kart8.tex
 

\chapter{\lab{ch8}Numerical methods\index{numerical methods}}
{\it\Large{Ed Bueler}}
\vspace*{0.3in}
\section{\lab{sec8.1}Shallow flow}

comments 


  * My highest priority is to stick to the slogans I have in my Karthaus slides,
    first "focus on approximating ice flow" (i.e. my limited scope) and
    second "example numerical codes actually work".
    
  * Thus a high priority is the inclusion of about four 40 line (or so) Matlab
    codes right into the text.  And about 12 codes which can be grabbed from
    an online source.  Is this o.k.?  If not I really need to start over.
    It is hard to "capture" the spirit of my lectures without saying "here is
    a running code that actually does something and has no mysteries now".
  
  * My goal is 20--25 pages; my Karthaus lectures in 2009 and 2010 were three
    hours but brevity is a better goal.

  * I am assuming that this chapter is only quasi-independent of the rest,
    in the sense that I can be very brief in stating SIA and SSA models, and 
    can refer to conservation of mass and energy material elsewhere.

  * The headings and subheadings were chosen with some care but I hope they
    will be revisable.  I'm afraid the structure of my material depends in
    part on the content of chapters 1 (=flow of ice), 2 (=thermal structure),
    and 10 (=analytical models) at least.

  * Can I suggest putting analytical models before numerical methods (e.g.
    swap current 8 and 10)?  I don't plan on creating a strong dependence
    but I am happy for readers to see the "right" order.

\subsection{\lab{sec8.1.1} Slab-on-a-slope}
\subsection{\lab{sec8.1.2} Shallow ice approximation (SIA)}
  \subsection{\lab{sec8.1.3} Scope of this chapter}

 \section{\lab{sec8.2} Finite differences}
  
    \subsection{\lab{sec8.2.1} Heat equation}
     \subsection{\lab{sec8.2.2} Explicit schemes and instabilities}
     \subsection{\lab{sec8.2.3} Diffusion equations and SIA}
     \subsection{\lab{sec8.2.4} Adaptive time steps}
     \subsection{\lab{sec8.2.5} Exact solutions and verification}
     \subsection{\lab{sec8.2.6} Ice sheet models}
  
  \section{\lab{sec8.3} Solving the stress balance for ice shelves}
  
     \subsection{\lab{sec8.3.1} Shallow shelf approximation (SSA)}
     \subsection{\lab{sec8.3.2} Nonlinear iterations}
     \subsection{\lab{sec8.3.3} Linear algebraic systems}
     \subsection{\lab{sec8.3.4} Ice streams and the grounding line}

 \section{\lab{sec8.4} Improving flow models}

     \subsection{\lab{sec8.4.1} Mass continuity and kinematical equations}
     \subsection{\lab{sec8.4.2} Longitudinal averaging}
     \subsection{\lab{sec8.4.3} Hybrid models}
     \subsection{\lab{sec8.4.4} Less-shallow approximations, Stokes}
     \subsection{\lab{sec8.4.5} Temperature, melt, sliding}
     \subsection{\lab{sec8.4.6} Skills and tools}
     
 \section{\lab{sec8.5}Notes and references}

  

\bex

%1
\item \lab{q8.1} 

\eex

%\item \lab{q8.8}
%ch8

%chapter 9 
\setcounter{subsection}{0}
\reseteqnos{9}
\include{kart9}%ch9

%chapter 10 
\setcounter{subsection}{0}
\reseteqnos{10}
\include{kart10}%ch10

%chapter 11
\setcounter{subsection}{0}
\reseteqnos{11}
\include{kart11}%ch11

%chapter 12
\setcounter{subsection}{0}
\reseteqnos{12}
\include{kart12}%ch12

%chapter 13
\setcounter{subsection}{0}
\reseteqnos{13}
\include{kart13}%ch13

%chapter 14
\setcounter{subsection}{0}
\reseteqnos{14}
\include{kart14}%ch14

%chapter 15
\setcounter{subsection}{0}
\reseteqnos{15}
\include{kart15}%ch15

%chapter 16
\setcounter{subsection}{0}
\reseteqnos{16}
\include{kart16}%ch16

%chapter 17
\setcounter{subsection}{0}
\reseteqnos{17}
\include{kart17}%ch17

%chapter 18
\setcounter{subsection}{0}
\reseteqnos{18}
\include{kart18}%ch18

%chapter 19
\setcounter{subsection}{0}
\reseteqnos{19}
\include{kart19}%ch19

%chapter 20
\setcounter{subsection}{0}
\reseteqnos{20}
\include{kart20}%ch20

\resetchapno{A}
\setcounter{subsection}{0}
\renewcommand{\thesection}{A}
	\setcounter{section}{0}
\reseteqnos{A}
\resetfignos{A}
\include{kart_apA}%projects

\resetchapno{B}
\setcounter{subsection}{0}
\renewcommand{\thesection}{B}
	\setcounter{section}{0}
\reseteqnos{B}
\resetfignos{B}
\include{kart_apB}%excursions


\include{kartrefs}%
\addcontentsline{toc}{chapter}{\protect\numberline{}{Index}}

\printindex
\end{document}

